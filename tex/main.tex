% -------------------------------------------------------------------
% APA-Style thesis LaTeX Template
% AUTHOR: Álvaro "Gondolindrim" Volpato (alvaro.volpato@usp.br)
% VERSION: 1.1
% PAGE: http://github.com/Gondolindrim/apaThesis
% LICENSE: Creative-Commons Non-commercial Share-Alike
%--------------------------------------------------------------------

% This is an example of document using the apaThesis class file,
% for a thesis that fits the American Phychology Association standards, described in 

% American Psychology Association (2009). Publication Manual of the American Psychological Association, sixth edition.

% It is supposed to be used as a template for a masters or PhD thesis. It was written in VIM and it contains
% folding data (the three "{") in the text. To enable this folding, type in vim:

% :setfoldmethod=marker

\documentclass{apaThesis}

% -----------------------------------------------
% (1) DOCUMENT DATA {{{1
% -----------------------------------------------

% Title and author
\title{APA Thesis LaTeX template}
\author{Álvaro Augusto Volpato}
\place{São Carlos, Brazil}
\writingdate{March 2019}
\affiliation{%
  University of São Paulo
  \par
  São Carlos School of Engineering
  \par
  Department of Electrical and Computer Engineering}
\advisor{Professor Dr. Luís Fernando Costa Alberto}
\preamble{Template document developed for thesis documents fulfilling the guidelines of the Publication Manual of the American Psychology Association, Sixth Edition}

% uniLogoWidth is the width of the university/institute logo at the title page
%\renewcommand*\uniLogoWidth{0.2\textwidth}

% \chapterpath and \appendixpath are commands that change the paths of *.tex files of chapters and appendixes. By default these paths are /tex/chapters and /tex/appendixes; you can change these paths as you like through \chapterspath and \appendixpath
%\chapterpath{./chapters}
%\appendixpath{./appendixes}
%\graphicspath{../images}
\begin{document}

% PRINTING TITLE PAGE
\printtitlepage

% PRINTING FRONT MATTER
\printfrontmatter

% -----------------------------------------------
% (2) SECOND FRONT MATTER {{{1
% -----------------------------------------------

% SECOND FRONT MATTER IN SECOND LANGUAGE
% If this second frontmatter is not needed, comment lines 53-62.
% The anotherfrontmatter command takes seven arguments and prints a second frontmatter with them, without however changing the \title, \author, \place, \affiliation, \advisor keys which were defined previously.
% For its code, see line 544 of the class file apaThesis.cls .
\anotherfrontmatter
{Template de tese no estilo APA}%
{\theauthor}%
{São Carlos, Brasil}%
{Março de 2019}%
{Universidade de São Paulo \par %
Escola de Engenharia de São Carlos \par%
Departamento de Engenharia Elétrica e de Computação}%
{Professor Luís Fernando Costa Alberto}%
{Documento padronizado (``template'') desenvolvido para uso em teses segundo o Manual de Publicações da Associação Americana de Psicologia, Sexta Edição.}

% -----------------------------------------------
% (3) CATALOGRAPHIC CARD {{{1
% -----------------------------------------------

% CATALOGRAPHIC CARD
% I was not able to make a program for this as it can differ wildly between countries -- some won't even need one; therefore, unfortunately I was not able to make a macro or a custom command for this and it has to be edited manually. This is a widely used template, but please check if it fulfills the requirements of your institution and edit however needed.
\thispagestyle{empty}
\vspace*{\fill}
\begin{center}
\noindent\fbox{%
	\parbox{0.9\textwidth}{%
		\leftskip0.05\textwidth
		\vspace*{1cm}
		\parbox{0.8\textwidth}{%
		\ttfamily\footnotesize
		Volpato, Álvaro Augusto

		\begingroup\addtolength{\parindent}{2em}

			\thetitle\hphantom{ } / \theauthor; advisor Luís Fernando Costa Alberto - São Carlos 2019.			

			\thelastpage\hphantom{} p.\\

			Dissertation (Master's Degree - Graduate Program in Electrical Engineering) -- São Carlos School of Engineering, University of São Paulo - Brazil, 2019\\

			1. LaTeX. 2. American Psychology Association. I. Alberto, Luís Fernando Costa, advisor. II. Title.

		\endgroup
		\vspace*{1cm}
    }%
}
}
\end{center}

\cleardoublepage

% -----------------------------------------------
% (4) ABSTRACTS {{{1
% -----------------------------------------------
%\pagestyle{plain}

% In order to handle many abstracts in various languages, the "newabstract" environment was created, defined in lines 460-467 of the classe file "apaThesis.cls".
% Each time a new abstract is added, all that is needed is:
% \begin{newabstract}{<abstract name in that language>} 
% 	ABSTRACT CONTENT \\
% \noindent
% \textbf{<Keyword name in that language>}: keywords
% \end{newabstract}
% \newpage	

% (4.1) IN ENGLISH
\begin{newabstract}{Abstract}
	This is the abstract in english. \\
\noindent
\textbf{Keywords}: keyword 1, keyword 2, keyword 3 ...
\end{newabstract}

% (4.2) AUS DEUTSCH
\begin{newabstract}{Zusammenfassung}
Zusammenfassung aus Deutsch. \\

\noindent
\textbf{Stichwörter}: Stichwort 1, Stichwort 2,...
\end{newabstract}

% (4.3) EM PORTUGUÊS
\begin{newabstract}{Resumo}
	Resumo em português. \\
\noindent
\textbf{Palavras-chave}: palavra-chave 1, palavra-chave 2, ...
\end{newabstract}

% -----------------------------------------------
% (5) LISTS OF FIGURES, TABLES, SYMBOLS AND ACRONYMS {{{1
% -----------------------------------------------
% Figures and tables will be automatically added as they appear in text.

% LIST OF FIGURES
\listoffigures
\cleardoublepage

% LIST OF TABLES
\listoftables
\cleardoublepage

% -----------------------------------------------
% LIST OF ABBRV. AND SYMBOLS
% -----------------------------------------------
% This piece of code centralizes the LoA and LoS names, putting them in a HUGE and boldfaced font.
% I was not able to insert this inside a \newenvironment, because this snippet messes with the
% @makeschapter macro, which parameter #1 conflicts with the parameter number definition of the
% newenvironment giving a "Illegal number definition" error. 

\begingroup
	\makeatletter
		\def\chapter{\cleardoublepage\secdef\@chapter\@schapter}
		\def\@makeschapterhead#1{{\center\HUGE\sffamily\bfseries #1\par\nobreak\vskip 10\p@\vspace*{5mm} }}
	\makeatother

% The LoA and LoS can be removed by commenting their respective lines if not needed.
% Defining acronyms
\begin{acronyms}
	\acronym{OMIB}{One Machine Infinite Bus System}
	\acronym{CG}{Centralized Generation}
	\acronym{DG}{Distributed Generation}
	\acronym{RPGTs}{Renewable Power Generation Technologies}
	\acronym{MASs}{Multi-Agent Systems}
	\acronym{EPSs}{Electric Power Systems}
	\acronym{ESSs}{Energy Storage Systems}
	\acronym{MPPT}{Maximum Power Point Tracking}
	\acronym{PV}{Photovoltaic}
	\acronym{PBD}{Pinning-based Droop}
	\acronym{SM}{Synchronous Machine}
	\acronym{OAM}{One-Axis Model}
	\acronym{ICMPPT}{Incremental Conductance Maximum Power Point Tracking}
\end{acronyms}

\begin{listofsymbols}
	\item[$ \Gamma $] Gamma greek letter
	\item[$ \Lambda $] Lambda greek letter
	\item[$ \zeta $] Lowercase Zeta greek letter
	\item[$ \in $] Set theory belonging/contained in relation
	\item[$|\cdot|$] Complex absolute value
	\item[$\lVert \cdot \rVert$] Complex vector or matrix euclidian norm
\end{listofsymbols}

\endgroup

% -----------------------------------------------
% (6) TABLE OF CONTENTS {{{1
% -----------------------------------------------

\tableofcontents*
\cleardoublepage

% Redefine plain style
\pagestyle{fancy}

% -----------------------------------------------
% (7) EPIGRAPH {{{1
% -----------------------------------------------

\begin{newepigraph}
``An optimist will tell you the glass is half-full.
				
A pessimist will say it's half-empty.

An engineer will tell you the glass is twice the size it needs to be.''

\hfill --- Unknown Author
\end{newepigraph}

%% -----------------------------------------------
% (8) TEXT BODY {{{1
% -----------------------------------------------

% Begin page number at one and start arabic page numbering
\begintextbody

\addchapter{introduction.tex}
\addchapter{figures_and_tables.tex}

%% ----------------------------------------------
% (9) BIBLIOGRAPHY {{{1
% -----------------------------------------------
\bibliography{refs}

% -----------------------------------------------
% (10) APPENDIXES {{{1
% -----------------------------------------------

\part*{Appendixes}
\appendix

\addappendix{appendixA.tex}
\addappendix{appendixB.tex}

\end{document}
