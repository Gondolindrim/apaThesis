% ----------------------------------------------
\chapter{Figures and tables}
% ----------------------------------------------

	This chapter verses on how to add figures, tables and scripts to your document.

	Adding those features is pretty straightforward as they are already formatted to the APA style. The only thing you should take care when adding these elements is the element size in relation to the page, that is, if the figure or table does not extend past the margins.

	For figures, this is generally a tradeoff between having legible text in the figure and having enough page space, meaning that you should adjust the image until it is readable and legible and its size is correct.

% ----------------------------------------------
	\section{Adding figures}
% ----------------------------------------------

	To add a figure to the document, use syntax in listing \ref{lst:addingFigure}.

\begin{lstlisting}[caption = {Adding a figure}, label = {lst:addingFigure}, style = prettyListing, language = tex]
\begin{figure}[h]
	\centering
	\includegraphics[width = <desired width>]{<Figure file location>}
	\caption{<Figure caption>}
	\label{<Human-readable label>}
\end{figure}
\end{lstlisting}

	This procedure is highly customizable, according to the parameters of the various environments and commands used.

	 The {\ttfamily\small figure} environment makes a floating figure that is adjusted by the \LaTeX\ engine . The {\ttfamily\small [h]} option defines that the figure should be placed preferably where it is placed in the source text, but that does not mean it will necessarily be placed there. For more placing options, see \href{https://tex.stackexchange.com/questions/35125/how-to-use-the-placement-options-t-h-with-figures}{this StackExchange page}.

	{\ttfamily\small \textbackslash centering } forces figure centering; {\ttfamily\small \textbackslash caption\{\} } is the figure caption that explains or defines it; {\ttfamily\small \textbackslash label\{\} } is the human-readable label that you use to reference the image in the text.

	The command {\ttfamily\small \textbackslash includegraphics[]\{\} } is the command to insert the figure. This command defines the size of the inserted image; for example, using {\ttfamily\small \textbackslash width = 5cm} will add a five-centimeter-wide figure that is also vertically scaled. See listing \ref{lst:5cmEESC}; figure \ref{fig:5cmEESC} shows the output of that code.

\begin{lstlisting}[caption = {5cm-wide EESC logo}, label = {lst:5cmEESC}, style = prettyListing, language = tex]
\begin{figure}[h]
	\centering
	\includegraphics[width = 5cm]{../images/uniLogo.pdf}
	\caption{Centimeter-wide EESC logo}
	\label{fig:5cmEESC}
\end{figure}
\end{lstlisting}

\begin{figure}[h]
	\centering
	\includegraphics[width = 5cm]{../images/uniLogo.pdf}
	\caption{Five-centimeter-wide EESC logo.}
	\label{fig:5cmEESC}
\end{figure}

	To make it easier for you to scale graphics in relation to the text column width, this length is given in \LaTeX\ by the command {\ttfamily\small \textbackslash textwidth}; for instance, to insert an image that has width of a quarter the text width, use {\ttfamily\small width = 0.25\textbackslash textwidth}, as in listing \ref{lst:quarterTextEESC}. Figure \ref{fig:quarterTextEESC} shows the output of such code.

\begin{lstlisting}[caption = {EESC logo with quarter the text column width}, label = {lst:quarterTextEESC}, style = prettyListing, language = tex]
\begin{figure}[h]
	\centering
	\includegraphics[width = 0.25\textwidth]{../images/uniLogo.pdf}
	\caption{EESC logo with a quarter the text column width}
	\label{fig:quarterTextEESC}
\end{figure}
\end{lstlisting}

\begin{figure}[h]
	\centering
	\includegraphics[width = 0.25\textwidth]{../images/uniLogo.pdf}
	\caption{EESC logo with a quarter the text column width.}
	\label{fig:quarterTextEESC}
\end{figure}
